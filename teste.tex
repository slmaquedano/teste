\documentclass[a4paper,12pt]{report}
\usepackage[brazil,brazilian]{babel}%habilita hifenizações	
\usepackage[T1]{fontenc}            %permite acentuação
\usepackage[utf8]{inputenc}         %idem anterior 
%\usepackage[margin= 2.5cm]{geometry}%define o tamanho das margens
\usepackage[hmargin=2cm,vmargin=2.0cm,bmargin=2cm]{geometry}	
\usepackage{times}
\usepackage{graphicx}               %inclusão de figuras
\usepackage{multirow}               %permite juntar linhas de uma tabela
%\usepackage{wallpaper}
\usepackage{array}
\pagestyle{empty}
\usepackage{draftwatermark}
\SetWatermarkText{\includegraphics [scale=1.5, angle=315]{marcadagua2.png}}


\begin{document}
%\ThisTileWallPaper{1.0\paperwidth}{1.0\paperheight}{logo.png}

\begin{center}
	\begin{tabular}{m{3.0cm}m{10.0cm}m{3.0cm}}
		\includegraphics[width=3.0cm]{uemp.png} &
		\begin{center}
			\large UNIVERSIDADE ESTADUAL DE MARINGÁ
	
			\large PROGRAMA DE EDUCAÇÃO TUTORIAL          
		
			\large GRUPO PET-FÍSICA						   
		\end{center} 							   &
		
		\includegraphics[width=3.0cm]{petp.png} \\
	\end{tabular}
\end{center}
\vspace{0.7cm}

\begin{center}
	{\large {\bf Ata 495/18}}
\end{center}

\vspace{0.4cm}

Ao quinto dia do mês de abril de dois mil e dezoito, 
às oito horas e trinta minutos, nas dependências da sala 014 do bloco F-67 na Universidade Estadual de Maringá, deu-se início a mais uma reunião do grupo PET-Física. O PETiano João se ausentou com justificativa e o PETiano Luiz atrasou sem justificativa. A reunião foi constituída pelos seguintes itens de pauta: 
1) Ata;
2) T.I.;
3) Eventos;
4) Jornal A Resistência;
5) Jornal Mural; 
6) CinePET; 
7) Seminário; 
8) Linhas de Pesquisa; 
9) PDV; 
10) Assuntos gerais. 
De início, 
1) a Ata 494/18 do PETiano Leandro foi lida e aprovada com retificações, já as atas referentes aos meses de serão impressas para a próxima reunião. 
Em seguida, 
2) a comissão irá elaborar uma nova capa para o {\itshape Facebook} e uma assinatura para o email.  
3) Acerca do ônibus para o SulPET, concordou-se com a despesa de duzentos e trinta reais a ser custeada por cada grupo, com a sugestão de aumentar este valor e,consequentemente, diminuir o desembolso do caixa da UniPET, dinheiro este que poderia ser empregado em aluguéis futuros -ônibus para o EnaPET- ou próximos eventos; a Organização do SulPET encaminhou as correções para os artigos submetidos    
os quais deverão ser atualizados e enviados até a próxima sexta-feira; o PET-Engenharia Química fez o convite para a discussão dos GDTs para o SulPET que ocorrerá no dia vinte e, sobre o Intercâmbio PETiano com o mesmo grupo, participarão os PETianos Pedro, Gabrielly e Isabela, sendo os mesmos responsáveis pelo apadrinhamento dos visitantes do PET-EQ, foi também preenchido o horário das 18h10 as 18h30  para apresentação do seminário (19).
4)para a correçãp do Jornal A Resistência organizou-se a seguinte ordem:

\begin{table} [h!]
\centering
\begin{tabular}{|	c	|	c	|}
\hline
%$\Delta$ s' & $f$   \\ \hline
Leandro e fernanda		&  Rebeca e joão    	\\ \hline
Rebeca e João           & Adão e Gabrielly  	\\ \hline
Adão e Gabrielly        & Pedro e Ellen  		\\ \hline
Pedro e Ellen           & Leandro e Fernanda 	\\ \hline
Isabela					& Lucas 				\\ \hline
Lucas					& Guilherme 			\\ \hline
Guilherme 				& Jessica 				\\ \hline
Jessica 				& Isabela				\\ \hline
Gabriel					& Luiz 					\\ \hline
\end{tabular}
\end{table}

\noindent Sendo que o texto do PETiano Gabriel não será corrigido em decorrência de ser uma notícia e o texto do PETiano Luiz enviado até
Sobre 5) o tutor enviará o jornal corrigido até terça-feira (10) e o mesmo deverá ser colado até a próxima reunião.
Em 6) os PETianos Adão, Rebeca, Gabrielly e Isabela se ausentaram com justificativa, os PETianos Fernanda, Jessica, Ellen e Pedro compareceram com justificado atraso e o PETiano Leandro se ausentou sem justificativa; Acertou-se que as apresentações do Grupo ocorreriam em primeira instância na sala 021 do bloco G56; O certificado para o filme {\it 2001 Uma Odisseia no Espaço} será emitido até a próxima reunião; a sessão referente ao mês de maio apresentará o filme {\it Ele Está de Volta} onde o PETiano Guilherme será responsável por providenciar o filme, o PETiano Adão pela sua exibição e ambos responáveis pela discussão; no mês de junho será passado o filme {\it Jurassic Park} e em julho o filme {\it De Volta para o Futuro}.  
Em seguida 7) o seminário a ser apresentado na aula de Oficina I na quarta-feira (11) às 19h30min ocorrerá no anfiteatro do Bolco D67, foram confirmadas as apresentações sobre telescópios e utensílios astronômicos, a serem apresentados pelos PETianos Pedro e Fernanda, concomitante a apresentação sobre o planetário.
Acerca de 8) as linhas de pesquisa  Cromatografia e Astronomia nas Escolas aguardam respostas do Colégio de Aplicações Pedagógicas, o Planetário  será detetizado no presente dia e lavado na segunda-feira (09), foi criado um grupo no {\it WhatsApp} para o grupo de Lógica e o PETiano Leandro participará do mesmo, o grupo de Apresentações didáticas não apresentou resultados.
Em 9) O PETiano Guilherme aguarda respostas da Professora Ângela sobre as monitorias no Colégio Instituto de Educação de Maringá; o PETiano Pedro enviará um arquivo para lista de presença no grupo do {\it Facebook} 
Por último 9) o PETiano Adão providenciará um calendário para antações, antes imcubido ao PETiano João; a limpeza para a próxima reunião será feita pelos PETianos Leandro e Ellen, a Comissão de Assuntos Internos providenciará caneta azul para o quadro e tintas das demais cores; O PETiano Guilherme levará a impressora 3D para conserto.
Nada mais havendo a tratar a reunião encerrou-se às 10h58min e eu, Lucas Maquedano da Silva, lavrei a presente ata que após lida e aprovada, segue assinada pelo Grupo. 
 
 
\vfill 

\begin{center}
	\begin{tabular}{m{7cm}m{7cm}}
		\begin{flushleft}
		\begin{center} 
			..................................................................
						
			 Tutor Prof. Dr. Marcos Cesar Danhoni Neves		
		\end{center}
		\end{flushleft}
		&
		\begin{flushright}
		\begin{center}
			.......................................................
						
			 PETiano Responsável
		\end{center}		
		\end{flushright}
	\end{tabular}
\end{center}

\end{document}




%_____________________________________________________
%\begin{center}
%{\large UNIVERSIDADE ESTADUAL DE MARINGÁ}\\[0.2cm]
%{\large PROGRAMA DE EDUCAÇÃO TUTORIAL}\\[0.2cm]
%{\large GRUPO PET-FÍSICA}\\[1.50cm]
%{\large \bf	\underline{ATA 123}}\\[0.7cm] 
%\end{center}
	
%\begin{figure}[!t]
%\flushleft
%\includegraphics[width=3cm]{uem.png} 
%\flushright 
%\includegraphics[width=3cm]{pet.png}
%\end{figure}
 
% asdfg asdfg asdfg\\[3.0cm]
 
%\begin{center} 
%\flushleft
%{\tiny \bf TUTOR PROF. DR. MARCOS CESAR DANHONI NEVES}
%\flushright
%{\tiny \bf PETIANO RESPONSÁVEL}
%\end{center} 

%\end{document}


